\documentclass{article}

\usepackage[utf8]{inputenc}
\usepackage[T1]{fontenc}
\usepackage[francais]{babel}

\begin{document}

\begin{itemize}

\item Ce document\cite{ref1}, est utile pour travailler sur la génération de
playlists dynamiques. L'article est centré sur l'utilisation des programmes de
radio pour créer des playlists y ressemblant.
\item L'article\cite{ref2} permet de spécialiser les metadonnnés musicales en
les liant entre elles. Permettrait une meilleure précision dans la génération
basée sur les styles musicaux principalement.
\item L'analyse proposé par le document\cite{ref3} serait utile pour trouver
l'émotion associée au fichier sonore.
\item L'article\cite{ref4} propose une confluence des tags utilisateurs pour
former un folklore musical, une taxonomie des fichiers musicaux. Très utile pour
classer des morceaux avec un facteur social.
\item Le sujet\cite{ref5} traite de l'analyse des musiques polyphoniques,
pratique pour l'analyse de fichier pour construire nos playlists.
\item L'article\cite{ref6} traite d'une expérience montrant les différences
entre les playlists basées sur des tags utilisateurs et celles basées sur les
métadonnées musicales. Utile pour le choix d'une de ces deux approches.
\item Le document\cite{ref7} contient des informations très utiles sur le million
song dataset.
\item Pour la partie visualisation de la playlist, le document\cite{ref8}
apporte des approches intéressantes.
\item Encore un document\cite{ref9} sur l'analyse du son.
\item Un État de l'art de la reconaissance d'émotions\cite{ref10}.
\item Un apercu des astuces de générations de playlists de qualité\cite{ref11}.

\end{itemize}

\bibliographystyle{plain}
\bibliography{memoire}

\end{document}