\documentclass{article}

\usepackage[utf8]{inputenc}
\usepackage[T1]{fontenc}
\usepackage[francais]{babel}

\title{Bibliographie commentée}
\author{Lucas Vauzelle \and Baptiste Fumaroli \and Jean Pecqueur \and Thomas Pizon}
\date{\today}

\begin{document}

\maketitle

\begin{itemize}

\item Musicovery
Musicovery est un site web qui propose une sélection de morceaux construite
autour de plusieurs types de paramètres qui peuvent être combinés pour affiner
la recherche.

En effet, le site propose une recherche par artiste ou par tag, ainsi qu'une
recherche par genre, et propose au sein d'une sélection de places les morceaux
autour de deux axes d'humeur~: sombre/positif et énergique/calme.

Le sité propose également quelques options plus sociales et basées sur la
sélection des utilisateurs.

http://musicovery.com/

\item Spotibot
Spotibot est un site internet qui permet de créer une playlist d'après un nom
d'artiste donné par l'utilisateur ou basée sur son profil Last.fm. La lecture
est donc effectuée sur Spotify.

Une fois que l'utilisateur a validé sa recherche, Spotibot lui renvoie une
playlist d'un nombre de morceaux préalablement défini, en fonction de la
similarité avec l'artiste demandé(via les tags, et via l'historique des écoutes
du compte Last.fm).

La playlist peut aussi être exportée pour être lue directement sur Spotify par
simple Drag & Drop.

http://www.spotibot.com/

\item Sourcetone
Sourcetone permet de classer des musiques d'une bibliothèque musicale en
différentes catégories, pour avoir des playlists correspondant à l'humeur et/ou
à l'activité du moment. Par exemple, avoir une playlist entraînante pour faire
du sport, ou une playlist plus calme pour se reposer le soir.

Sourcetone possède également une application radio, que l'utilisateur peut
paramétrer selon ses envies (musique rythmée ou non, joyeuse ou non, de quelle
année, etc.).

L'application se base sur les évaluations et les avis de l'utilisateur,
«~apprend~» à mieux le connaître, afin de lui proposer des musiques de plus en
plus adaptées selon son humeur.

http://www.sourcetone.com/

\end{itemize}

\end{document}